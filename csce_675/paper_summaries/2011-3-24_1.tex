\documentclass[12pt, onecolumn]{IEEEtran}
\newtheorem{theorem}{Theorem}
\newtheorem{lemma}{Lemma}
\newtheorem{definition}{Definition}

\usepackage{algorithm}
\usepackage{algorithmic}
\usepackage{graphicx}

\begin{document}
% \title{Daily readings -  10 September 2009}
% \author{\IEEEauthorblockN{Krishna Kamath}\\
% \IEEEauthorblockA{
% kykamath@cse.tamu.edu}}
% \maketitle

\textbf{\large \begin{center}Paper Summary
\end{center} }
\begin{center}Krishna Y.Kamath (UIN: 119001316)\\
kykamath@cs.tamu.edu\\
Date: 24 March, 2011
\end{center}
        
\section{Paper Citation Information}
\begin{itemize}
  \item Title: The very small world of the well-connected \cite{shi:wellconnected}
  \item Authors: Xiaolin Shi, Matthew Bonner, Lada A. Adamic and Gilbert
  Anna C.
  \item Publication details: HT '08: Proceedings of the nineteenth ACM conference on Hypertext and hypermedia
\end{itemize}

\section{Summary}

\subsection{Problem Statement}
Most communication online happens through the social connections people have
online. These communications result in the formation of social graphs which are
huge in size. To understand the flow of information, design efficient
algorithms to mine these graphs, we have to study the statistical and graph
properties of the entire network. The properties that we need to understand
might be degree distribution, clustering properties and the features of the
network governing its growth and evolution. But the large size of social
networks makes the calculation of these properties difficult. 
\\
One of the solutions to this problem has been identifying important or
critical nodes in these networks and evaluating properties of these reduced
graphs. But, there is no principled way study of how to reduce these graphs
and ensuring the reduced graphs represent the quality of the actual graph.
Having a accurate way to reduce a huge graph is important because it there
are several online social networks that can be reduced without loss of
quality. The reduced graph can be stored efficiently. The paper tries
to build a simple principled method to build the reduced graph. Reduced graph
is also called as synopsis of the actual graph.
\\
\subsection{Proposed Solution}
The paper build the synopsis of the graph using important vertices, using which
most of the properties of the actual graph can be identified without losing the
actual graphs quality. 
\\
They  paper first defines steps to identify the important nodes
in the network. It defines 4 parameters based on which importance of nodes can
be defined. Degree of a node is number of other nodes it is connected to.
Betweeness of a node is the number of pair of nodes that have a path through the
node. Closeness of a node measures the distance from other graphs. Pagerank is
a method of measuring the importance of a node in comparison to its neighbors.
The paper uses 4 datasets- erdos-reyni graph, buddyzoo graph, TREC and world
wide web graph, to compare their analytical results empirically.
\\
The paper first analyzes the actual network for the 4 properties mentioned
before. They then build a sub-graph of the important nodes and calculates
these properties for this graph. The paper them compares the properties of the
actual graph with the synopsis graph, built using important nodes. The paper then compares the quality of the compressed graph. It then analyzes 2
heuristic algorithms KeepAll and KeepOne. Theoretical analysis of 
\\
\subsection{Results}
After studying the graph they observe that a large graph can be reduced to a
sub-graph of important nodes. They observe that different importance measures
generate different sub graphs of varying density and topology. They identify
different nodes as important, but inspite of these difference they agree in
certain parts that are important for developing synopsis is a graph. Hence,
they conclude that sub-graphs can be used to determine the properties of the
large online social networks.
\\
\section{Critique}
\subsection{Strenghts}
The area of online social networks is very interesting and the paper deals with
the problem of understanding these networks. The huge size of social networks
makes it difficult to analyze these graphs. The paper tries to build a sub-graph
of the larger graph that reflects the properties of the larger graph. The
problem they are trying to solve is important, hence the paper's problem and
solution is relevant to its field of research.
\\
The paper is very well written. The authors clearly define the problem they are
are trying to address. They define, in advance, the various importance metrics
use in the paper, like degree distribution, page rank, betweeness and
closeness. They then perform theoretical analysis of their solution. The
experiments they perform clearly show that their technique meets the goals
desired by the paper.
\\
\subsection{Weakness}
One of the places that, I thought, the paper could improve was in their
experimental setup. The data-set that the paper uses to test its results are not
very diverse. The size of the data-set is small compared to many actual social
networks. Hence, testing their results on these larger graphs might have helped
us understanding their techniques better.

\section{Future work}
The results produced by the paper are interesting. Their work can be taken
ahead in different ways. We can take their solution and apply that on larger
and more real graphs and see if we observe results similar to this paper.
Understanding if these techniques also work on larger graphs is important to
the real world use of their techniques. 
\\
\bibliographystyle{IEEEtran}
\bibliography{IEEEabrv,675}
 
\end{document}
