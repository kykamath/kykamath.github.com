\documentclass[12pt, onecolumn]{IEEEtran}
\newtheorem{theorem}{Theorem}
\newtheorem{lemma}{Lemma}
\newtheorem{definition}{Definition}

\usepackage{algorithm}
\usepackage{algorithmic}
\usepackage{graphicx}

\begin{document}
% \title{Daily readings -  10 September 2009}
% \author{\IEEEauthorblockN{Krishna Kamath}\\
% \IEEEauthorblockA{
% kykamath@cse.tamu.edu}}
% \maketitle

\textbf{\large \begin{center}Paper Summary
\end{center} }
\begin{center}Krishna Y.Kamath (UIN: 119001316)\\
kykamath@cs.tamu.edu\\
Date: $24^{th}$ March, 2011
\end{center}
        
\section{Paper Citation Information}
\begin{itemize}
  \item Using interactive multi-media for teaching and learning object oriented
  software design \cite{choi:teaching}
  \item Authors: Sun-Hea Choi and Sandra Cairncross
  \item Publication details: ITiCSE '01: Proceedings of the 6th annual
  conference on Innovation and technology in computer science education
\end{itemize}

\section{Summary}

\subsection{Problem Statement}
Many first year students are exposed to programming for the first time.
Programming is an abstract concept and students who are introduced to it for the
first time have difficulties understanding it. Universities use verbal mode to
teach programming and explaining programming concepts verbally is difficult.
These difficulties discourage students to continuing studying programming. This
difficulty was observed in several Universities in UK. The paper deals with the
methods to improve teaching of programming concepts using multi-media
techniques.
\\
\subsection{Proposed Solution}
The paper first tries to understand what learning is and based on this it builds
its solution. It believes reasoning and reflection to be important part of
learning. Actual learning happens when students reason about what they have
learned and then use it to do something constructive with their current
knowledge. They refer the Maye's learning framework, into three steps. The first is
conceptualization, where a student is exposed to knowledge. This is followed by
construction, where use of knowledge gained during conceptualization is used. The
third and final step is  dialogue, where the student refines his knowledge
through dialogue and discussion. Fowler and Mayes modify Maye's framework to add
dialogue at every stage of learning. They introduce clarification and
confirmation during conceptualization,  collaboration during construction phase
and identification during dialogue.

The paper makes use of these definitions to design its solution that uses
multimedia. It uses multimedia like video sound and animation to enhance the
experience of learning. One of the problems with traditional approach was,
presenting information in one format, i.e verbal, now using multimedia, they can
present information in more than one format, allowing them to explain the
concepts better. They can now use simulation and visualization to explain the
abstract concepts better, which cannot be explained verbally. Their solution
basically has two parts. The first one was a resource based material that can
be used in the conceptualization stage, when the student in trying to learn the
concept. The next stage is task-based solution, that can be used by the user
during construction stage of learning.

\subsection{Results}
To test test their application, the asked a bunch of students in the university
to use it. The multimedia solution helped students at Brunel and Napier to 
learn programming. They found students, with prior knowledge of multimedia were
able to better use the solution and hence, learn programming better. Prior
perception of what multi-media is effected the way the tool was used by the
instructors. The perception among instructors was that a multi-media tool was
something independent and students would be able to learn it themselves. On
the negative side they found out that some students were afraid to try anything
new with it and felt that the tool wasn't assisting them but increasing their
workload. But, they observed that, with time users got used to these solutions
and started liking it better. They were happy with visualization, since it
helped them understand abstract concepts better. By the end of their study they
found that 71\% of students at Brunel and 93\% of students at Napier gave
positive review for their solution.
\\
\section{Critique}
\subsection{Strenghts}
The problem of building tools to improve learning process is very
interesting. They identify the difficulties students face with learning,
identify techniques that can help learning and build solution according to
those techniques. Hence, the problem and the techniques used to solve the
problem are relevant.
\\
The results show that students though initially found difficult to use the
solution, they eventually liked it. It shows the their solution met its
intended goal.

\subsection{Weakness}
One of the places that, I thought, the paper could improve was in their
experimental setup. The data-set that the paper uses to test its results are not
very diverse. The size of the data-set is small compared to many actual social
networks. Hence, testing their results on these larger graphs might have helped
us understanding their techniques better.
\\
\section{Future work}
The paper could improve in their implementation details. They don't explain
the details of their implementation, like the tools they used, the architecture
of the solution etc. Giving those details would have made their solution much
more clear.
\\
\bibliographystyle{IEEEtran}
\bibliography{IEEEabrv,675}
 
\end{document}