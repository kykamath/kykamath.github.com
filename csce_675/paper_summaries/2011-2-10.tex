\documentclass[12pt, onecolumn]{IEEEtran}
\newtheorem{theorem}{Theorem}
\newtheorem{lemma}{Lemma}
\newtheorem{definition}{Definition}

\usepackage{algorithm}
\usepackage{algorithmic}
\usepackage{graphicx}

\begin{document}
% \title{Daily readings -  10 September 2009}
% \author{\IEEEauthorblockN{Krishna Kamath}\\
% \IEEEauthorblockA{
% kykamath@cse.tamu.edu}}
% \maketitle

\textbf{\large \begin{center}Paper Summary
\end{center} }
\begin{center}Krishna Y.Kamath (UIN: 119001316)\\
kykamath@cs.tamu.edu\\
Date: $10^{rd}$ February, 2010
\end{center}
        
\section{Paper Citation Information}
\begin{itemize}
  \item Distributed Interactive Learning Environment \cite{807725}
  \item Authors: Jim X. Chen, Christopher J Dede, Xiaodong Fu and Yonggao Yang
  \item Publication details: Proceedings of the 2008 Conference on Human Factors in Computing
               Systems, CHI 2008, 2008, Florence, Italy, April 5-10, 2008.
               Pages 883-892.
\end{itemize}

\section{Summary}

\subsection{Problem Statement}
Understanding abstract scientific concepts is difficult. It becomes much harder
for students in school to understand these concepts. For example, students
understand swirling waterforms present in swimming pools, but find it hard to
understand how Navier-Stokes algorithm models this. The number of parameters used
in these kind of equations add to the complexity. The paper tries to develop
strategies and approaches to implement an synthetic learning environment. Such an
environment will help students to interact with each other in a better way and
collaborate on exercises. Learning things in a collaborative way helps students
understand the abstract concepts better. Hence, the paper deals with the problem
of implementing a realistic, scalable solution to help students.
\\
\subsection{Proposed Solution}
In this paper the authors construct a virtual class room to help distance
education using computational steering and interactive visualization. Using
existing technologies like Distributed Interactive Simulation(DIS) they
implement their research results in a network infrastructure. DIS is a field of
networked simulation research and technology, which helps in real-time display
and synchronization. 

DIS is a multi-cast application that requires low latency and high reliability.
But, currently it uses UDP protocol to transfer messages which is a best-effort
protocol and hence does not ensure reliability. Other protocols like transfer
protocol provide such guarantees, but they are currently not used in DIS. The
main problem in implementing this application is the mutually conflicting
requirements of real-time performance and reliability. Hence, the authors
implement a version of protocol that satisfies these conflicting requirements.
They also ensure that their solution gives high performance and scalability.

Another important requirement is ensuring synchronization. Since the
application is collaborative in nature, it is important that the actions users
take are synchronized. Synchronization also ensures fast and accurate
simulations.  To provide this feature the paper uses a uniform time scale
proportional to the clock-time and variable time-slicing to synchronize physical
models.

\subsection{Results}
The paper presented a DIS environment learning and training purposes. It
provided an interface, that students, instructors and experts, who are all
geographically separated, can use to collaborate. Their application gives a
real-time virtual learning environment that links these users. The application
is implemented such that all manipulations and numerical simulations are
synchronized in DIS. Using the interface learners can discuss common experience
with other learners and learn their experiences with manipulating parameters
and perspectives. This helps them to master the theories they are trying to
learn. In addition to this instructors and subject matter experts can
demonstrate abstract concepts using graphics animations and other
multimedia related techniques. Thus the tool provides a real-time collaborative
environment.
\\
\section{Critique}
\subsection{Strenghts}
The problem of providing a real time collaborative environment is very
interesting. The problem is important because, such a solution can be used in
distance learning. The paper does a good work of identifying, potential problem
such a solution may face and gives satisfactory solutions to overcome those
problems. The paper is organised well. They first identify the problems, give
solutions to these problems and then give details about the implementations.
\\
\subsection{Weakness}
The evaluation can be improved if the authors could test the performance of
actual collaborative environments. In the paper the authors give no detail
about any user study conducted on their application. Such a study may help us
understand the effectiveness of the solution. We will be able to better
understand the scalability and performance of the application.
\\
\section{Future work}
The paper could improve in their implementation details. They don't explain
the details of their implementation, like the tools they used, the architecture
of the solution etc. Giving those details would have made their solution much
more clear.
\\
\bibliographystyle{IEEEtran}
\bibliography{IEEEabrv,675}
 
\end{document}