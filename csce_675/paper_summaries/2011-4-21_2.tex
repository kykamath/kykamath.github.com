\documentclass[12pt, onecolumn]{IEEEtran}
\newtheorem{theorem}{Theorem}
\newtheorem{lemma}{Lemma}
\newtheorem{definition}{Definition}

\usepackage{algorithm}
\usepackage{algorithmic}
\usepackage{graphicx}

\begin{document}
% \title{Daily readings -  10 September 2009}
% \author{\IEEEauthorblockN{Krishna Kamath}\\
% \IEEEauthorblockA{
% kykamath@cse.tamu.edu}}
% \maketitle

\textbf{\large \begin{center}Paper Summary
\end{center} }
\begin{center}Krishna Y.Kamath (UIN: 119001316)\\
kykamath@cs.tamu.edu\\
Date: $21^{st}$ April, 2011
\end{center}
        
\section{Paper Citation Information}
\begin{itemize}
  \item Where did the researchers go?: supporting social navigation at a large
  academic \cite{Farzan:2008:go}
  \item Authors: Rosta Farzan and Peter Brusilovsky
  \item Publication details: HT ’08: Proceedings of the nineteenth ACM
  conference on Hypertext and hypermedia. New York, NY, USA: ACM, 2008, pp.
  203, 212

\end{itemize}

\section{Summary}

\subsection{Problem Statement}
The paper tries to deal with the problem of information overload. A common
approach taken when there is problem of information overload, is to follow some
other user who we trust and who has gone through the information path. The
problem that paper deals with is similar to this. It tries to understand the role
played by social navigation and social search technologies in the context of
conference attendance planning. Academic conference generally have a multiple
parallel sessions with large number of papers in each session. Hence, deciding
which paper to see is a difficult decision. The paper tries to use to collective
knowledge of the community and social navigation techniques to solve this
problem. They develop an application called Conference Navigator using these
techniques to solve the problem of information overload in conferences.

\subsection{Proposed Solution}
Traditionally there have been some problems observed with collecting feedback
from users. The application depends on collecting information from the user.
Hence, it deals with these problems as well. The first problem is the amount of
time users have to spend giving the actual feedback. This action must be
explicitly performed and hence effects with the natural order of their
activities. To solve this problem the application tries to introduce activities
for users that provide reliable indication of users interest. The second
problem faced by this process in the issue of privacy of feedback given by the
users. This issue is resolved by tracking the feedback at a community level
rather that at the individual level. 

To access the system the user first has to select a community. If he doesn't find
a suitable community he can create another new community. The application has 2
modules. A schedule browser, that any user can use. It provides a listing of all
the tracks and papers. A user can browse through the conference and decide on
the conference he wants to attend. Another module is the Personal schedule
planner module. This can be used by only registered users. This can be used to
schedule the papers one is  interested in attending. While browsing a user can
perform 3 functions. (i). Simple visiting: In this the user just browses
through all the tracks and the papers. (ii). Annotating: In this function a
user can annotate the papers he is interested in with tags. (iii). Scheduling:
In this function a user can add a paper to his schedule. The application uses
scheduling and annotating to build the ``wisdom'' of the communities. This can
then be used to make decisions about which conferences to attend.

Social navigation support is provided by augmenting links to papers presented
to user during searching and browsing. The returned articles are marked with
icons that show the popularity of the article within the community. It uses
information like, the number of people from your community that are going to
attend the paper, their annotations etc.

\subsection{Results}
The application was evaluated in the E-Learn 2007 conference series organized by
AACE. They evaluated the system on a limited number of users and noticed that the
users who used the system, took decisions to attend a particular paper or not
depending on the annotations that were returned next to the article. The
conference was divided into 15 parallel tracks with a total of 150 papers. The
users were asked to use the system by distributing flyers. They were also give a
questionnaire to obtain feedback on the utility of the system. They then
performed application log analysis to test its performance. Based on their
investigation they determined that using the application the users were able to
identify useful papers based on the recommendations provided by the system.

\section{Critique}
\subsection{Strenghts}
The problem of solving the information overload at conferences is interesting.
The problem is important because, such a solution can be used in solving other
information overload scenarios. The paper does a good work of identifying,
potential problem such a solution may face and gives satisfactory solutions to
overcome those problems. The paper is organized well. They first identify the
problems, give solutions to these problems and then give details about the
implementations.

\subsection{Weakness}
One of the problems with the paper is that it is too specific to the problem of
information overload in conferences and not general to the problem of information
overload. It also does not tell us if their approach can be used in any other
scenario other that this. Also no details about the scalability and performance
of the application is given.

\section{Future work}
The paper could be useful if they can suggest ways in which their current work
can be used in other general cases of information overload. They also can imporve
the paper by providing the more details of their implementation. They don't
explain the details of their implementation, like the tools they used, the
architecture of the solution etc. Giving those details would have made their
solution much more clear.

\bibliographystyle{IEEEtran}
\bibliography{IEEEabrv,675}
 
\end{document}
