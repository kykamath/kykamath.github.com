\documentclass[12pt, onecolumn]{IEEEtran}
\newtheorem{theorem}{Theorem}
\newtheorem{lemma}{Lemma}
\newtheorem{definition}{Definition}

\usepackage{algorithm}
\usepackage{algorithmic}
\usepackage{graphicx}

\begin{document}
\textbf{\large \begin{center}CSCE-675 Digital Libraries: Daily Readings
\end{center} }
\begin{center}Krishna Y.Kamath (UIN: 119001316)\\
kykamath@cs.tamu.edu\\
Date: $21$ April, 2011
\end{center}

\section{Setting the Foundations of Digital Libraries: The DELOS Manifesto \cite{ross:2007:dl}}
\noindent Details:
\begin{itemize}
  \item Year: 2007
  \item Type: Article in D-lib Magazine
  \item Affiliations: Italian National Research Council (CNR)
\end{itemize}
\medskip
Take-away points:
\begin{itemize}
  \item This paper identifies the need for a robust model of Digital Libraries encapsulating the richness of these perspectives is required.
  \item Hence, they create the draft for The Digital Library Manifesto, the aim of which is to set the foundations and identify the cornerstone concepts within the universe of Digital Libraries, facilitating the integration of research results and proposing better ways of developing appropriate systems.
  \item It exploits the collective understanding that has been acquired, over more than a decade, on Digital Libraries by European research groups active in the Digital Library field, both within DELOS and outside, as well as by other groups around the world.
\end{itemize}
\bigskip\bigskip
%%%%%%%%%%%%%%%%%%%%%%%%%%%%%%%%%%%%%%%%%%%%%%%%%%%%%%%%%%%%%%%%%%%%%%%%%%%%%%%

\section{Where Do We Go From Here? \cite{lynch:2005:dl}}
\noindent Details:
\begin{itemize}
  \item Year: 2005
  \item Type: Article in D-lib Magazine
  \item Affiliations: Coalition for Networked Information
\end{itemize}
\medskip
Take-away points:
\begin{itemize}
  \item The field of digital libraries has always been poorly-defined, a "discipline" of amorphous borders and crossroads, but also of atavistic resonance and unreasonable inspiration.
  \item The paper looks at the various challenges the DL can take on in the future.
  \item The author feels the next generation of digital libraries would work on connecting and integrating digital libraries with broader individual, group and societal activities.
\end{itemize}
\bigskip\bigskip
%%%%%%%%%%%%%%%%%%%%%%%%%%%%%%%%%%%%%%%%%%%%%%%%%%%%%%%%%%%%%%%%%%%%%%%%%%%%%%%


\bibliographystyle{IEEEtran}
\bibliography{675}
\end{document}
