\documentclass[12pt, onecolumn]{IEEEtran}
\newtheorem{theorem}{Theorem}
\newtheorem{lemma}{Lemma}
\newtheorem{definition}{Definition}

\usepackage{algorithm}
\usepackage{algorithmic}
\usepackage{graphicx}

\begin{document}
\textbf{\large \begin{center}CSCE-675 Digital Libraries: Daily Readings
\end{center} }
\begin{center}Krishna Y.Kamath (UIN: 119001316)\\
kykamath@cs.tamu.edu\\
Date: $1^{st}$ February 2011
\end{center}

\section{Variations2: retrieving and using music in an academic setting
\cite{dunn:2006:variations}}

\noindent Details:
\begin{itemize}
  \item Year: 2006
  \item Type: Communications of the ACM - Music information retrieval
  \item Affiliations: Indiana University, Bloomington, IN
\end{itemize}
\medskip
Take-away points:
\begin{itemize}
  \item Variations 2 is a Java client-server application, that was developed to
  replace the original Variations in 2005.
  \item It allows students to search and retrieve music.
  \item It allows students and instructors to add timepoints between sections
  creating bubbles representing multi-level musical structures.
\end{itemize}
	
\noindent Questions to discuss
\begin{itemize}
  \item The application uses very simple features of the musical records. The
  retrieval can be improved by using digital signals from music records.
  \item Interfaces specific to expertise of different users, like one for
  general users and another for expert users.
  \item The project doesn't mention Pandora genome project, which sounds to me
  is very close to what the paper is trying to do.
\end{itemize}
\bigskip\bigskip
%%%%%%%%%%%%%%%%%%%%%%%%%%%%%%%%%%%%%%%%%%%%%%%%%%%%%%%%%%%%%%%%%%%%%%%%%%%%%%%

\section{Supporting elementary-age children's searching and browsing: Design and
evaluation using the international children's digital library
\cite{hutchinson:2007:children}}

\noindent Details:
\begin{itemize}
  \item Year: 2007
  \item Type: Journal of the American Society for Information Science and Technology
  \item Affiliations: Human�Computer Interaction Lab, Department of Computer Science and College of Information Studies, University of Maryland, College Park, MD
\end{itemize}
\medskip
Take-away points:
\begin{itemize}
  \item The paper describes a search and browzing tool for
  elementary-age children (ages 6-11).
  \item They in particular want to overcome the challenges of boolean retrieval
  and heirarchy based browzing provided current search engines, as these
  features tends to be challenging to children.
  \item Their research shows that a �at category structure, where
	only leaf-level categories are available and can be viewed
	simultaneously provides much better results.
\end{itemize}
	
\noindent Questions to discuss
\begin{itemize}
  \item I don't agree with the claim of the authors use that children find
  writing queries difficult.
  \item The research should have been evaluated with a more diverse demography.
  \item Is this system currently in use.
\end{itemize}
\bigskip\bigskip
%%%%%%%%%%%%%%%%%%%%%%%%%%%%%%%%%%%%%%%%%%%%%%%%%%%%%%%%%%%%%%%%%%%%%%%%%%%%%%%


\section{Visual Knowledge: Textual Iconography of the Quixote, a Hypertextual Archive
\cite{furuta2005:quixote}}

\noindent Details:
\begin{itemize}
  \item Year: 2005
  \item Type: Oxford Journals Humanities Literary and Linguistic Computing
  \item Affiliations: Texas A\&M University, College Station, TX
\end{itemize}
\medskip
Take-away points:
\begin{itemize}
  \item The paper describes a system to visualize Quixote.
  \item They give a specification of a comprehensive taxonomy of the episodes,
  adventures, themes and characters in the  Quixote.
  \item They also plan to develop a tool to compare, juxtapose and collage
  related images from several editions, artists, etc., as part of our research
  to create new approaches and techniques to display images for analysis, beyond
  browsing and searching.
\end{itemize}
	
\noindent Questions to discuss
\begin{itemize}
  \item What is the current state of this project?
\end{itemize}
\bigskip\bigskip
%%%%%%%%%%%%%%%%%%%%%%%%%%%%%%%%%%%%%%%%%%%%%%%%%%%%%%%%%%%%%%%%%%%%%%%%%%%%%%%

\section{A multilingual approach to technical manuscripts: 16th and 17th-century Portuguese shipbuilding treatises
\cite{monroy:2007:ship}}

\noindent Details:
\begin{itemize}
  \item Year: 2007
  \item Type: JCDL '07 Proceedings of the 7th ACM/IEEE-CS joint conference on Digital libraries
  \item Affiliations: Texas A\&M University, College Station, TX
\end{itemize}
\medskip
Take-away points:
\begin{itemize}
  \item In this paper the authors describe a scalable approach  and a
  multilingual web-based interface for enabling a group of scholars to edit a
  glossary of nautical terms in multiple languages.
  \item Shipbuilding treatises are ancient technical manuals that describe 
  properties of the wood and materials used (ship components), as  well as the
  steps to be followed in the construction of a composite object (ship).
  \item Multilingual documents are source materials that researchers in 
	other disciplines need to access for their scholarly work. 
	Therefore, the authors expect that our approach can help other digital 
	libraries� repositories with multilingual documents.
\end{itemize}
	
\noindent Questions to discuss
\begin{itemize}
  \item What is the current state of this project?
  \item 
\end{itemize}
\bigskip\bigskip
%%%%%%%%%%%%%%%%%%%%%%%%%%%%%%%%%%%%%%%%%%%%%%%%%%%%%%%%%%%%%%%%%%%%%%%%%%%%%%%

\bibliographystyle{IEEEtran}
\bibliography{675}

\end{document}