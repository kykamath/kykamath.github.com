\documentclass[12pt, onecolumn]{IEEEtran}
\newtheorem{theorem}{Theorem}
\newtheorem{lemma}{Lemma}
\newtheorem{definition}{Definition}

\usepackage{algorithm}
\usepackage{algorithmic}
\usepackage{graphicx}

\begin{document}
\textbf{\large \begin{center}CSCE-675 Digital Libraries: Daily Readings
\end{center} }
\begin{center}Krishna Y.Kamath (UIN: 119001316)\\
kykamath@cs.tamu.edu\\
Date: $20^{th}$ January 2011
\end{center}

\section{As We May Think.
\cite{aswemaythink1945}}

\noindent Details:
\begin{itemize}
  \item Year: 1945
  \item Type: Article in The Atlantic
  \item Affiliations: MIT 
\end{itemize}
\medskip
Take-away points:
\begin{itemize}
  \item The author points out several key areas where the advances made during
  the war can be used.
  \item He suggests the knowledge acquired during the war years should be used
  in developing technologies that will help people to access knowledge gained
  during previous years.
  \item In photography he talks about advancements that could be made like
  clicking pictures that are available immediately, compression of images, etc.
\end{itemize}
	
\noindent Questions to discuss
\begin{itemize}
  \item Have there been similar papers written after this?
  \item What are the kind of technologies one would think of if he was to write
  a paper similar to this?
\end{itemize}
\bigskip\bigskip
%%%%%%%%%%%%%%%%%%%%%%%%%%%%%%%%%%%%%%%%%%%%%%%%%%%%%%%%%%%%%%%%%%%%%%%%%%%%%%%

\section{An Open Architecture for a Digital Library System and a Plan for its
Development. \cite{openframework}}
\noindent Details:
\begin{itemize}
  \item Year: 1988
  \item Type: Article in The Digital Library Project, Volume 1: The World of
  Knowbots. (Washington, DC: Corporation for National Research Initiatives)
  \item Affiliations: Chairman, CEO and President of the Corporation for
  National Research Initiatives (CNRI); Vice President and Chief Internet
  Evangelist Google.
\end{itemize}
\medskip
Take-away points:
\begin{itemize}
  \item The authors present an architecture for digital library systems.
  \item It is designed to allow organistations to share information with other
  organistaions.
  \item They describe the various components that would be needed to support a
  digital library system.
\end{itemize}
	
\noindent Questions to discuss
\begin{itemize}
  \item This looks like a proposal. Did they build this system? 
  \item Was this paper the reason behind the authors being called Fathers of
  Internet?
  \item Does items generated by online users like tweets, status messages,
  comments etc be considered as a part of digital library system?
\end{itemize}
\bigskip\bigskip
%%%%%%%%%%%%%%%%%%%%%%%%%%%%%%%%%%%%%%%%%%%%%%%%%%%%%%%%%%%%%%%%%%%%%%%%%%%%%%%

\section{NSF/DARPA/NASA Digital Libraries Initiative A Program Manager's
Perspective \cite{dlprogram}} \noindent Details:
\begin{itemize}
  \item Year: 1998
  \item Type: Article in D-Lib Magazine
  \item Affiliations: Program Director: Special Projects Digital Libraries Initiative
\end{itemize}
\medskip
Take-away points:
\begin{itemize}
  \item The article gives a description of various initiatives in Digital
  libraries funded by NSF, DARPA and other government organisations.
  \item It uses the demand for high quality content and ease of access as the
  motivation that will drive the funding and development of digital libraries.
  \item It also gives a definition for digital libraries
\end{itemize}
	
\noindent Questions to discuss
\begin{itemize}
  \item Can we include the items generated by online users like tweets,
  status messages, comments etc be considered as a part of digital library
  system? How should we change this definition?
  \item How has the field changed in the 12 years since this article was
  written?
\end{itemize}
\bigskip\bigskip
%%%%%%%%%%%%%%%%%%%%%%%%%%%%%%%%%%%%%%%%%%%%%%%%%%%%%%%%%%%%%%%%%%%%%%%%%%%%%%%

\section{A National Digital Library for Science, Mathematics, Engineering, and
Technology Education \cite{dlstem}} \noindent Details:
\begin{itemize}
  \item Year: 1998
  \item Type: Article in D-Lib Magazine
  \item Affiliations: Division of Undergraduate Education, National Science
  Foundation; Mathematics Department Montana State University
\end{itemize}
\medskip
Take-away points:
\begin{itemize}
  \item Describes how Internet is enabling people to contribute to online
  knowledge.
  \item Gives scenarious where users use digital libraries for their work
  \item Gives details of architecture that has been developed to support digital
  libraries.
\end{itemize}
	
\noindent Questions to discuss
\begin{itemize}
  \item What is the current state of this project?
  \item Do we need to build systems specific to these resources or does web
  serve as a platform in itself which users can use?
\end{itemize}
\bigskip\bigskip
%%%%%%%%%%%%%%%%%%%%%%%%%%%%%%%%%%%%%%%%%%%%%%%%%%%%%%%%%%%%%%%%%%%%%%%%%%%%%%%

\bibliographystyle{IEEEtran}
\bibliography{675}

\end{document}