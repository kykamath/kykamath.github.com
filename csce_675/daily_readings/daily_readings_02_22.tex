\documentclass[12pt, onecolumn]{IEEEtran}
\newtheorem{theorem}{Theorem}
\newtheorem{lemma}{Lemma}
\newtheorem{definition}{Definition}

\usepackage{algorithm}
\usepackage{algorithmic}
\usepackage{graphicx}

\begin{document}
\textbf{\large \begin{center}CSCE-675 Digital Libraries: Daily Readings
\end{center} }
\begin{center}Krishna Y.Kamath (UIN: 119001316)\\
kykamath@cs.tamu.edu\\
Date: $22^{nd}$ February 2011
\end{center}

\section{Why Digitize?
\cite{abby:1999:digitize}}

\noindent Details:
\begin{itemize}
  \item Year: 1999
  \item Type: Article in CLIR
  \item Affiliations: Council on Library and Information Resources
\end{itemize}
\medskip
Take-away points: 
\begin{itemize}
  \item The author found that digitization often raises expectations
of benefits, cost reductions, and efficiencies that can be illusory and,
if not viewed realistically, have the potential to put at risk the collections
and services libraries have provided for decades.
\item The author suggests that the universities must first analyze if the cost
of digitization provides enough benefits.
\item The benefits of making an underused collection more accessible
should be viewed in conjunction with other factors such as compatibility with
other digital resources and the collection’s intrinsic intellectual value.
\end{itemize}
	
\noindent Questions to discuss
\begin{itemize}
  \item What are the copyright issues associated with digitization?
  \item How does Google's effort contribute in digitization?
\end{itemize}
\bigskip\bigskip
%%%%%%%%%%%%%%%%%%%%%%%%%%%%%%%%%%%%%%%%%%%%%%%%%%%%%%%%%%%%%%%%%%%%%%%%%%%%%%%

\section{Digital Imaging Tutorial}
\noindent Details:
\begin{itemize}
  \item Year: 2000
  \item Type: Article in Cornell Library
  \item Affiliations: Cornell University Library/ Research Department
\end{itemize}
\medskip
Take-away points: 
\begin{itemize}
  \item This tutorial provides introduction to the use of digital
  imaging to convert and make accessible cultural heritage materials.
  \item It also describes some concepts advocated by Cornell library like, the
  value of benchmarking requirements before undertaking a digital initiative.
\end{itemize}
	
\noindent Questions to discuss
\begin{itemize}
  \item What are the copyright issues associated with digitization?
  \item How does Google's effort contribute in digitization?
\end{itemize}
\bigskip\bigskip
%%%%%%%%%%%%%%%%%%%%%%%%%%%%%%%%%%%%%%%%%%%%%%%%%%%%%%%%%%%%%%%%%%%%%%%%%%%%%%%



\bibliographystyle{IEEEtran}
\bibliography{675}

\end{document}