\documentclass[12pt, onecolumn]{IEEEtran}
\newtheorem{theorem}{Theorem}
\newtheorem{lemma}{Lemma}
\newtheorem{definition}{Definition}

\usepackage{algorithm}
\usepackage{algorithmic}
\usepackage{graphicx}

\begin{document}
\textbf{\large \begin{center}CSCE-675 Digital Libraries: Daily Readings
\end{center} }
\begin{center}Krishna Y.Kamath (UIN: 119001316)\\
kykamath@cs.tamu.edu\\
Date: $19$ April, 2011
\end{center}

\section{MyLifeBits: fulfilling the Memex vision \cite{Gemmell:2002:MFM}}
\noindent Details:
\begin{itemize}
  \item Year: 2002
  \item Type: Proceedings of the tenth ACM international conference on Multimedia
  \item Affiliations: Microsoft Research, San Francisco, CA
\end{itemize}
\medskip
Take-away points:
\begin{itemize}
  \item MyLifeBits is a project to fulfill the Memex vision first posited
by Vannevar Bush in 1945.
  \item  It is a system for storing all of one’s
digital media, including documents, images, sounds, and videos.
  \item It is built on four principles: (1) collections and search must replace hierarchy for organization (2) many visualizations should be supported (3) annotations are critical to non-text media and must be made easy, and (4) authoring should be via transclusion.
\end{itemize}
\bigskip\bigskip
%%%%%%%%%%%%%%%%%%%%%%%%%%%%%%%%%%%%%%%%%%%%%%%%%%%%%%%%%%%%%%%%%%%%%%%%%%%%%%%

\section{MyLifeBits: a personal database for everything \cite{Gemmell:2006:MPD}}
\noindent Details:
\begin{itemize}
  \item Year: 2006
  \item Type: MyLifeBits: a personal database for everything
  \item Affiliations: Microsoft Research in Redmond, WA
\end{itemize}
\medskip
Take-away points:
\begin{itemize}
  \item This paper describes the progress for the project whose goals included under- standing the effort to digitize a lifetime of legacy content and the elimination of paper as a permanent storage medium. 
  \item They base their project on memex.
  \item They observe that irrespective of their attempts, the amount of information a user stores increases as more and more things get digitized.
\end{itemize}
\bigskip\bigskip
%%%%%%%%%%%%%%%%%%%%%%%%%%%%%%%%%%%%%%%%%%%%%%%%%%%%%%%%%%%%%%%%%%%%%%%%%%%%%%%

\section{Building a research library for the history of the web \cite{Arms:2006:BRL}}
\noindent Details:
\begin{itemize}
  \item Year: 2006
  \item Type: Proceedings of the 6th ACM/IEEE-CS joint conference on Digital libraries
  \item Affiliations: Cornell University Ithaca, NY
\end{itemize}
\medskip 
Take-away points:
\begin{itemize}
  \item This paper describes the building of a research library for studying the Web, especially research on how the structure and content of the Web change over time.
  \item The library is particularly aimed at supporting social scientists for whom the Web is both a fascinating social phenomenon and a mirror on society.
  \item The technical challenges that they observed fall into two categories: high-performance computing to transfer and manage the very large amounts of data, and human-computer interfaces that empower research by non-computer specialists.


\end{itemize}
\bigskip\bigskip
%%%%%%%%%%%%%%%%%%%%%%%%%%%%%%%%%%%%%%%%%%%%%%%%%%%%%%%%%%%%%%%%%%%%%%%%%%%%%%%


\bibliographystyle{IEEEtran}
\bibliography{675}
\end{document}
