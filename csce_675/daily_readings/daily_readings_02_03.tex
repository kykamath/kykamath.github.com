\documentclass[12pt, onecolumn]{IEEEtran}
\newtheorem{theorem}{Theorem}
\newtheorem{lemma}{Lemma}
\newtheorem{definition}{Definition}

\usepackage{algorithm}
\usepackage{algorithmic}
\usepackage{graphicx}

\begin{document}
\textbf{\large \begin{center}CSCE-675 Digital Libraries: Daily Readings
\end{center} }
\begin{center}Krishna Y.Kamath (UIN: 119001316)\\
kykamath@cs.tamu.edu\\
Date: $3^{rd}$ February 2011
\end{center}

\section{Cross-cultural usability of the library metaphor
\cite{Duncker:2002:culture}}

\noindent Details:
\begin{itemize}
  \item Year: 2002
  \item Type: Proceedings of the 2nd ACM/IEEE-CS joint conference on Digital libraries
  \item Affiliations: Middlesex University, London, UK
\end{itemize}
\medskip
Take-away points: 
\begin{itemize}
  \item The 
paper presents an investigation of the cross-cultural use and 
usability of such metaphors by studying the library metaphor of 
digital libraries in the cultural context of the Maori, the indigenous 
population of New Zealand.
\item  Their study examines 
relevant features of the Maori culture, their form of knowledge 
transfer and their use of physical and digital libraries.
\item The paper points out why and when the library metaphor fails Maori 
and other indigenous users, and indicates how this knowledge can 
contribute to the improvement of future designs.  
\end{itemize}
	
\noindent Questions to discuss
\begin{itemize}
  \item One of the points made by the authors was that, Maori felt more
  confortable learning from computers than interacting the Pakeha. I found this
  point interesting.
  \item I found it interesting how metaphors are used to display icons. 
  \item The use of metaphors results in reverse learning of concepts used in
  metaphors.
\end{itemize}
\bigskip\bigskip
%%%%%%%%%%%%%%%%%%%%%%%%%%%%%%%%%%%%%%%%%%%%%%%%%%%%%%%%%%%%%%%%%%%%%%%%%%%%%%%

\section{The travails of visually impaired web travellers
\cite{Goble:2000:travails}}

\noindent Details:
\begin{itemize}
  \item Year: 2000
  \item Type: HYPERTEXT '00 Proceedings of the eleventh ACM on Hypertext and hypermedia
  \item Affiliations: Information Management Group, Department of Computer
  Science, University of Manchester, Oxford Road, Manchester M13 9PL, UK
\end{itemize}
\medskip
Take-away points:
\begin{itemize}
  \item This paper proposes the inclusion of travel and mobility in 
the usability metrics of web design.
\item This paper presents the ground work for 
including travel into web design and usability metrics by 
presenting a framework for identifying travel objects and 
registering them as either cues to aid travel or obstacles that 
hinder travel for visually impaired users.
\item Knowledge of the differences in travel between 
visually impaired and sighted people will enable the model 
to be used in assisting the design of better user agents and 
web content for visually impaired and other users.  
\end{itemize}
	
\noindent Questions to discuss
\begin{itemize}
  \item How accurate is the evaluation.
  \item Shouldn't the sample se for evaluation be much bigger.
\end{itemize}
\bigskip\bigskip
%%%%%%%%%%%%%%%%%%%%%%%%%%%%%%%%%%%%%%%%%%%%%%%%%%%%%%%%%%%%%%%%%%%%%%%%%%%%%%%

\section{Interpretation of web page layouts by blind users
\cite{Francisco-Revilla:2010:blind}}

\noindent Details:
\begin{itemize}
  \item Year: 2010
  \item Type: Proceedings of the 10th annual joint conference on Digital libraries
  \item Affiliations: The University of Texas, Austin, TX, USA
\end{itemize}
\medskip
Take-away points:
\begin{itemize}
  \item Currently, approaches that translate two-dimensional layouts to 
one-dimensional speech produce a very different user experience 
and loss of information.
\item To address this issue, this paper conducted a 
study of how blind people navigate and interpret layouts of news 
and shopping Web pages using current assistive technology.
\item Their
study revealed that blind people do not parse Web pages fully 
during their first visit, and that they can miss important parts. Their 
study also provided insights for improving assistive technologies.
\end{itemize}
	
\noindent Questions to discuss
\begin{itemize}
  \item How accurate is the evaluation.
  \item Shouldn't the sample set for evaluation be much bigger.
\end{itemize}
\bigskip\bigskip
%%%%%%%%%%%%%%%%%%%%%%%%%%%%%%%%%%%%%%%%%%%%%%%%%%%%%%%%%%%%%%%%%%%%%%%%%%%%%%%

\bibliographystyle{IEEEtran}
\bibliography{675}

\end{document}