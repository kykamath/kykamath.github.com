\documentclass[12pt, onecolumn]{IEEEtran}
\newtheorem{theorem}{Theorem}
\newtheorem{lemma}{Lemma}
\newtheorem{definition}{Definition}

\usepackage{algorithm}
\usepackage{algorithmic}
\usepackage{graphicx}

\begin{document}
\textbf{\large \begin{center}CSCE-675 Digital Libraries: Daily Readings
\end{center} }
\begin{center}Krishna Y.Kamath (UIN: 119001316)\\
kykamath@cs.tamu.edu\\
Date: $25^{th}$ January 2011
\end{center}

\section{The Perseus Project and Beyond: How Building a Digital Library Challenges the Humanities and Technology
\cite{crane:98:persus}}

\noindent Details:
\begin{itemize}
  \item Year: 1998
  \item Type: Article in D-Lib Magazine
  \item Affiliations: Editor-in-Chief, Associate Professor of Classics, Tufts University, Medford, Massachusetts
\end{itemize}
\medskip
Take-away points:
\begin{itemize}
  \item Perseus is a digital library for humanities materials related
  to Greco-Roman era.
  \item It describes the challenges a group of humaists face in building a
  digital library.
  \item It describes a generic framework for developing a digital library
  irrespective of the type of collection.
\end{itemize}
	
\noindent Questions to discuss
\begin{itemize}
  \item What is the current state of this project?
  \item What are other challenges a group wanting to build digital libraries
  would face?
  \item Are there any other projects similar to Perseus?
\end{itemize}
\bigskip\bigskip
%%%%%%%%%%%%%%%%%%%%%%%%%%%%%%%%%%%%%%%%%%%%%%%%%%%%%%%%%%%%%%%%%%%%%%%%%%%%%%%


\section{The Networked Digital Library of Theses and Dissertations: Changes in the University Community
\cite{fox:02:thesis}}

\noindent Details:
\begin{itemize}
  \item Year: 2002
  \item Type: Article in D-Lib Magazine
  \item Affiliations:  Director, Digital Library Research Laboratory (Virginia
  Tech)
\end{itemize}
\medskip
Take-away points:
\begin{itemize}
  \item On the first anniversary of funding by the U.S. Department of Education
  (FIPSE) for a National Digital Library of Theses and Dissertations, the
  article describes the progress it has made and the controversies it has
  generted.
  \item It describes the collaborated effort made by Virginia Tech, University
  of Waterloo and Darmstadt University of Technology.
  \item It describes the controversy related to copyrights, the project  has had
  to deal with.
\end{itemize}
	
\noindent Questions to discuss
\begin{itemize}
  \item What is the current state of this project?
  \item Examples of other domains, where similar opposition has been seen to
  adoption of a technology?
  \item How are thesis and dissertations currently shared?
\end{itemize}
\bigskip\bigskip
%%%%%%%%%%%%%%%%%%%%%%%%%%%%%%%%%%%%%%%%%%%%%%%%%%%%%%%%%%%%%%%%%%%%%%%%%%%%%%%


\section{The Digital Library: A Biography
\cite{greenstien:02:biography}}

\noindent Details:
\begin{itemize}
  \item Year: 2002
  \item Type: Article in D-Lib Magazine
  \item Affiliations:  Director of the California Digital Library 
\end{itemize}
\medskip
Take-away points:
\begin{itemize}
  \item This article presents librarians and library directors with experiences
  of other Universoties in going digital.
  \item It gives case studies from California Digital Library (CDL), Harvard
  University, Indiana University, New York University (NYU), the University of
  Michigan, and the University of Virginia.
  \item The study was based on the results of the survey and interviews.
\end{itemize}
	
\noindent Questions to discuss
\begin{itemize}
  \item Why would a big University not collaborate with smaller ones?
  \item Did Texas A&M face similar problems or were they different?
\end{itemize}
\bigskip\bigskip
%%%%%%%%%%%%%%%%%%%%%%%%%%%%%%%%%%%%%%%%%%%%%%%%%%%%%%%%%%%%%%%%%%%%%%%%%%%%%%%

\bibliographystyle{IEEEtran}
\bibliography{675}

\end{document}