\documentclass[12pt, onecolumn]{IEEEtran}
\newtheorem{theorem}{Theorem}
\newtheorem{lemma}{Lemma}
\newtheorem{definition}{Definition}

\usepackage{algorithm}
\usepackage{algorithmic}
\usepackage{graphicx}

\begin{document}
\textbf{\large \begin{center}CSCE-675 Digital Libraries: Daily Readings
\end{center} }
\begin{center}Krishna Y.Kamath (UIN: 119001316)\\
kykamath@cs.tamu.edu\\
Date: $7$ April, 2011
\end{center}

\section{Toward an ecology of hypertext annotation
\cite{Marshall_1998}}
\noindent Details:
\begin{itemize}
  \item Year: 1998
  \item Type: HYPERTEXT '98
  \item Affiliations: Microsoft Research
\end{itemize}
\medskip
Take-away points: 
\begin{itemize}
  \item This  paper  first  characterizes  annotation  according  to a set of  dimensions  to  situate  a long-term  study  of  a community  of  annotators. 
  \item Using  the results  of  the  study,  the  paper  explores  the  implications  of  annotative practice  for  hypertext  concepts  and for  the development  of an  ecology  of  hypertext  annotation,  in  which  consensus creates a reading  structure  from  an authorial  structure.
\end{itemize}
\bigskip\bigskip
%%%%%%%%%%%%%%%%%%%%%%%%%%%%%%%%%%%%%%%%%%%%%%%%%%%%%%%%%%%%%%%%%%%%%%%%%%%%%%%

\section{Effects of annotations on student readers and writers
\cite{Wolfe_2000}}
\noindent Details:
\begin{itemize}
  \item Year: 2000
  \item Type: Proceedings of the fifth ACM conference on Digital libraries
  \item Affiliations: University of Texas at Austin, Austin, TX
\end{itemize}
\medskip
Take-away points: 
\begin{itemize}
  \item This paper reports on a study of persuasive essays written by 123 undergraduates receiving primary source materials annotated in various ways. 
  \item Their findings indicate that annotations improve recall of emphasized items, influence how specific arguments in the source materials are perceived, decrease students' tendencies to unnecessarily summarize.
\item Using this study as a basis, the author discusses implications for the design and implementation of digitally annotated materials.
\end{itemize}
\bigskip\bigskip
%%%%%%%%%%%%%%%%%%%%%%%%%%%%%%%%%%%%%%%%%%%%%%%%%%%%%%%%%%%%%%%%%%%%%%%%%%%%%%%

\section{Annotation functionality for digital libraries supporting collaborative performance: an example of musical scores
\cite{Winget_2007}}
\noindent Details:
\begin{itemize}
  \item Year: 2007
  \item Type: Proceedings of the 7th ACM/IEEE-CS joint conference on Digital libraries
  \item Affiliations: University of Texas at Austin, Austin, TX
\end{itemize}
\medskip
Take-away points: 
\begin{itemize}
  \item This paper describes the findings of an ethnographic study that examined the annotation behaviors of musicians working with musical scores for the purpose of performance. 
  \item Annotations were found to be an important part of the rehearsal process, and specific annotation functionalities are recommended for future digital library development.
\end{itemize}
\bigskip\bigskip
%%%%%%%%%%%%%%%%%%%%%%%%%%%%%%%%%%%%%%%%%%%%%%%%%%%%%%%%%%%%%%%%%%%%%%%%%%%%%%%

\bibliographystyle{IEEEtran}
\bibliography{675}
\end{document}