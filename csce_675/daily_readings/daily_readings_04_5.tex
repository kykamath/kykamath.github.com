\documentclass[12pt, onecolumn]{IEEEtran}
\newtheorem{theorem}{Theorem}
\newtheorem{lemma}{Lemma}
\newtheorem{definition}{Definition}

\usepackage{algorithm}
\usepackage{algorithmic}
\usepackage{graphicx}

\begin{document}
\textbf{\large \begin{center}CSCE-675 Digital Libraries: Daily Readings
\end{center} }
\begin{center}Krishna Y.Kamath (UIN: 119001316)\\
kykamath@cs.tamu.edu\\
Date: $5$ April, 2011
\end{center}

\section{Streams, Structures, Spaces, Scenarios,
Societies (5S): A Formal Model for Digital
Libraries
\cite{marcos:5s}}
\noindent Details:
\begin{itemize}
  \item Year: 2004
  \item Type: ACM Trans. Inf. Syst
  \item Affiliations: Virginia Polytechnic Institute and State University
\end{itemize}
\medskip
Take-away points: 
\begin{itemize}
  \item This paper gives formal models for digital libraries.
  \item  The authors give fundamental abstractions of Streams, Structures, Spaces, Scenarios, and Societies (5S), which allows them to de�ne digital libraries rigorously.
  \item  The applicability, versatility, and unifying power of the 5S model are demonstrated through its use in three distinct applications: building and interpretation of a DL taxonomy, informal and formal analysis of case studies of digital libraries (NDLTD and OAI), and utilization as a formal basis for a DL description language.
\end{itemize}
\bigskip\bigskip
%%%%%%%%%%%%%%%%%%%%%%%%%%%%%%%%%%%%%%%%%%%%%%%%%%%%%%%%%%%%%%%%%%%%%%%%%%%%%%%

\bibliographystyle{IEEEtran}
\bibliography{675}

\end{document}