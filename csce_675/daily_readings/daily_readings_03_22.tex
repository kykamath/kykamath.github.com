\documentclass[12pt, onecolumn]{IEEEtran}
\newtheorem{theorem}{Theorem}
\newtheorem{lemma}{Lemma}
\newtheorem{definition}{Definition}

\usepackage{algorithm}
\usepackage{algorithmic}
\usepackage{graphicx}

\begin{document}
\textbf{\large \begin{center}CSCE-675 Digital Libraries: Daily Readings
\end{center} }
\begin{center}Krishna Y.Kamath (UIN: 119001316)\\
kykamath@cs.tamu.edu\\
Date: $22^{nd}$ February 2011
\end{center}

\section{Evaluating Personal Archiving Strategies for Internet-based
               Information
\cite{pa_catherine}}

\noindent Details:
\begin{itemize}
  \item Year: 2007
  \item Type: Journal Article in CoRR
  \item Affiliations: Microsoft Research 
\end{itemize}
\medskip
Take-away points: 
\begin{itemize}
  \item Internet-based personal digital belongings present different vulnerabilities than locally stored materials. The authors use responses to a survey of people who have recovered lost websites, in combination with supplementary interviews, to paint a fuller picture of current curatorial strategies and practices.
 \item They examine the type of information lost by the users and the reasons why they lost them.
 \item Their study reveals ways in which expectations of permanence and notification are violated and situations in which benign neglect has far greater consequences for the long-term fate of important digital assets. 
\end{itemize}
\bigskip\bigskip
%%%%%%%%%%%%%%%%%%%%%%%%%%%%%%%%%%%%%%%%%%%%%%%%%%%%%%%%%%%%%%%%%%%%%%%%%%%%%%%


\section{The Long Term Fate of Our Digital Belongings: Toward a Service Model for Personal Archives
\cite{long_catherine}}

\noindent Details:
\begin{itemize}
  \item Year: 2006
  \item Type: Proceedings of IS\&T Archiving 2006
  \item Affiliations: Microsoft Research
\end{itemize}
\medskip
Take-away points: 
\begin{itemize}
  \item The authors conducted a preliminary  field study to understand the current state of personal digital archiving in practice.
 \item Their  findings not only confirmed that experienced home computer users are creating, receiving, and finding an increasing number of digital belongings, but also that they have already lost irreplaceable digital artifacts such as photos, creative efforts, and  records. 
\item They found four environmental factors that further complicate archiving in consumer settings: the pervasive influence of malware; consumer reliance on ad hoc IT providers; an accretion of minor system and registry inconsistencies; and strong  consumer beliefs about the incorruptibility of digital forms, the reliability of digital technologies, and the social vulnerability of networked storage
\end{itemize}
\bigskip\bigskip
%%%%%%%%%%%%%%%%%%%%%%%%%%%%%%%%%%%%%%%%%%%%%%%%%%%%%%%%%%%%%%%%%%%%%%%%%%%%%%%

\section{Lazy preservation: reconstructing websites by crawling the crawlers
\cite{lazy_frank}}

\noindent Details:
\begin{itemize}
  \item Year: 2006
  \item Type: Proceedings of the 8th annual ACM international workshop on Web information and data management
  \item Affiliations:  Harding University
\end{itemize}
\medskip
Take-away points: 
\begin{itemize}
  \item The authors introduce �lazy preservation� � digital preservation performed as a result of the normal operation of web crawlers and caches.
\item They evaluate the e?ectiveness of lazy preservation by reconstructing 24 websites of varying sizes and composition using Warrick, a web-repository crawler.
\item They also measured the time required for web resources to be discovered and cached (10-103 days) as well as how long they remained in cache after deletion (7-61 days).
\end{itemize}
\bigskip\bigskip
%%%%%%%%%%%%%%%%%%%%%%%%%%%%%%%%%%%%%%%%%%%%%%%%%%%%%%%%%%%%%%%%%%%%%%%%%%%%%%%


\bibliographystyle{IEEEtran}
\bibliography{675}

\end{document}