\documentclass[12pt, onecolumn]{IEEEtran}
\newtheorem{theorem}{Theorem}
\newtheorem{lemma}{Lemma}
\newtheorem{definition}{Definition}

\usepackage{algorithm}
\usepackage{algorithmic}
\usepackage{graphicx}

\begin{document}
\textbf{\large \begin{center}CSCE-675 Digital Libraries: Daily Readings
\end{center} }
\begin{center}Krishna Y.Kamath (UIN: 119001316)\\
kykamath@cs.tamu.edu\\
Date: $22^{nd}$ February 2011
\end{center}

\section{What is a document?
\cite{doc_buckland}}
\noindent Details:
\begin{itemize}
  \item Year: 1997
  \item Type: Journal of the American Society for Information Science
  \item Affiliations: School of Information, Berkeley
\end{itemize}
\medskip
Take-away points: 
\begin{itemize}
  \item Ordinarily the word �document� denotes a textual record. The paper tries to define the term �document�.
  \item Increasingly sophisticated attempts to provide access to the rapidly growing quantity of available documents raised questions about what should be considered a �document.�
  \item The authors feel that with new digital technology the old questions are renewed and also old confusions between medium, message, and meaning.
\end{itemize}
\bigskip\bigskip
%%%%%%%%%%%%%%%%%%%%%%%%%%%%%%%%%%%%%%%%%%%%%%%%%%%%%%%%%%%%%%%%%%%%%%%%%%%%%%%

\section{Fixed or fluid?: document stability and new media
\cite{fixed_levy}}
\noindent Details:
\begin{itemize}
  \item Year: 1994
  \item Type: Proceedings of the 1994 ACM European conference on Hypermedia technology
  \item Affiliations: Xerox Palo Alto Research Center
\end{itemize}
\medskip
Take-away points: 
\begin{itemize}
  \item One of the crucial properties of documents through the ages has been their fixity. The paper tries to understand if this still holds using new media devices.
\item In this paper the authors challenges the assertion, that document in new media is fluid, arguing instead that all documents, regardless of medium, are fixed and fluid. 
\item The author first examines the fixity and fluidity of hypertext and then critiques Bolter's argument in Writing Space concerning the movement from �fixed to fluid.�
\end{itemize}
\bigskip\bigskip
%%%%%%%%%%%%%%%%%%%%%%%%%%%%%%%%%%%%%%%%%%%%%%%%%%%%%%%%%%%%%%%%%%%%%%%%%%%%%%%

\bibliographystyle{IEEEtran}
\bibliography{675}

\end{document}