\documentclass[12pt, onecolumn]{IEEEtran}
\newtheorem{theorem}{Theorem}
\newtheorem{lemma}{Lemma}
\newtheorem{definition}{Definition}

\usepackage{algorithm}
\usepackage{algorithmic}
\usepackage{graphicx}

\begin{document}
\textbf{\large \begin{center}CSCE-675 Digital Libraries: Daily Readings
\end{center} }
\begin{center}Krishna Y.Kamath (UIN: 119001316)\\
kykamath@cs.tamu.edu\\
Date: $8^{rd}$ February 2011
\end{center}

\section{The Reading Appliance Revolution
\cite{Schilit99thereading}}

\noindent Details:
\begin{itemize}
  \item Year: 1999
  \item Type: IBM Journal of Research and Development
  \item Affiliations: FX Palo Alto Laboratory, Inc., Palo Alto, CA, USA
\end{itemize}
\medskip
Take-away points: 
\begin{itemize}
  \item Reading appliances allow people to work on electronic documents much as
they would on paper. They therefore provide an alternative to the standard
“browse or search and then print” model of reading online. 
\item The authors describe how, by integrating a
wide variety of document activities, such as searching, organizing, and
skimming, and by allowing fluid movement among them, reading appliances
eliminate disruptive transitions between paper and digital media.
\item With XLibris they demonstrate, the combination of paperlike qualities
with digital capabilities allows us to augment these activities without
redefining them or radically changing the way people work.
\end{itemize}
	
\noindent Questions to discuss
\begin{itemize}
  \item Why did the kindle and ipad succed compared to other devices?
  \item Did the environment provided around the applicance, like distribution
  channel, have an impact?
  \item How important are the features of the product?
\end{itemize}
\bigskip\bigskip
%%%%%%%%%%%%%%%%%%%%%%%%%%%%%%%%%%%%%%%%%%%%%%%%%%%%%%%%%%%%%%%%%%%%%%%%%%%%%%%

\section{Reading in the office
\cite{Golovchinsky:2008:RO:1458412.1458420}}

\noindent Details:
\begin{itemize}
  \item Year: 2008
  \item Type: Proceeding of the 2008 ACM workshop on Research advances in large digital book repositories
  \item Affiliations: FX Palo Alto Laboratory, Inc., Palo Alto, CA, USA
\end{itemize}
\medskip
Take-away points: 
\begin{itemize}
  \item Advances in technology such as touch screens, light-weight 
high-power computers, and bi-stable displays have periodically 
renewed interest in online reading over the last twenty years, 
only to see that interest decline to a small early-adopter 
community.
\item  THe author argues that the true value 
of online reading lies in supporting activities beyond reading per 
se: activities such as annotation, reading and comparing multiple 
documents, transitions between reading, writing and retrieval, 
etc.
\item He concludes that the current hardware will be successful in the long 
term may depend on its abilities to address the reading needs of 
knowledge workers, not just leisure readers.
\end{itemize}
	
\noindent Questions to discuss
\begin{itemize}
  \item It is interesting how the annotations are used.
  \item I thought incorporating remote links as a part of the reader was very
  interesting.
\end{itemize}
\bigskip\bigskip
%%%%%%%%%%%%%%%%%%%%%%%%%%%%%%%%%%%%%%%%%%%%%%%%%%%%%%%%%%%%%%%%%%%%%%%%%%%%%%%

\section{InfoGallery: informative art services for physical library spaces
\cite{Gronbaek:2006:IIA:1141753.1141757}}

\noindent Details:
\begin{itemize}
  \item Year: 2006
  \item Type: JCDL '06 Proceedings of the 6th ACM/IEEE-CS joint conference on
  Digital libraries
  \item Affiliations: University of Aarhus, Aarhus, Denmark
\end{itemize}
\medskip
Take-away points: 
\begin{itemize}
  \item This paper describes InfoGallery, which is a web-based 
infrastructure for enriching the physical library space with 
informative art “exhibitions” of digital library material and other 
relevant information, such as RSS news streams, event 
announcements etc.
\item InfoGallery  presents information in an 
aesthetically attractive manner on a variety of surfaces in the 
library, including cylindrical displays and floors.
\item InfoGallery  presents information in an 
aesthetically attractive manner on a variety of surfaces in the 
library, including cylindrical displays and floors.
\end{itemize}
	
\noindent Questions to discuss
\begin{itemize}
  \item How important is the physical locations of the info galleries?
  \item How can we use a similar technology in another application?
\end{itemize}
\bigskip\bigskip
%%%%%%%%%%%%%%%%%%%%%%%%%%%%%%%%%%%%%%%%%%%%%%%%%%%%%%%%%%%%%%%%%%%%%%%%%%%%%%%

\section{Time as essence for photo browsing through personal digital libraries
\cite{Graham:2002:TEP:544220.544301}}

\noindent Details:
\begin{itemize}
  \item Year: 2002
  \item Type: JCDL '02 Proceedings of the 2nd ACM/IEEE-CS joint conference on Digital libraries
  \item Affiliations: Stanford University
\end{itemize}
\medskip
Take-away points: 
\begin{itemize}
  \item The authors develop two photo browsers for collections with thousands of
  time-stamped digital images. 
  \item The browsers exploit the timing information to structure the collections
  and to automatically generate meaningful summaries, and differ in how
  users navigate and view the structured collections.
  \item Their results show that exploiting the time dimension and appropriately
  summarizing collections can lead to significant improvements.
\end{itemize}
	
\noindent Questions to discuss
\begin{itemize}
  \item How can a similar product be use din a different domain?
\end{itemize}
\bigskip\bigskip
%%%%%%%%%%%%%%%%%%%%%%%%%%%%%%%%%%%%%%%%%%%%%%%%%%%%%%%%%%%%%%%%%%%%%%%%%%%%%%%

\bibliographystyle{IEEEtran}
\bibliography{675}

\end{document}